\documentclass[14pt, oneside]{altsu-report}

\worktype{Отчёт по практике на тему:}
\title{Разработка кроссплатформенного игрово проложения Bomberman на Unity}
\author{А.\,И.~Молчанова}
\groupnumber{5.205-2}
\GradebookNumber{1337}
\supervisor{И.\,А.~Шмаков}
\supervisordegree{к.ф.-м.н., доцент}
\ministry{Министерство науки и высшего образования}
\country{Российской Федерации}
\fulluniversityname{ФГБОУ ВО Алтайский государственный университет}
\institute{Институт цифровых технологий, электроники и физики}
\department{Кафедра вычислительной техники и электроники}
\departmentchief{В.\,В.~Пашнев}
\departmentchiefdegree{к.ф.-м.н., доцент}
\shortdepartment{ВТиЭ}
\abstractRU{Современная игровая индустрия является одной из самых динамично развивающихся отраслей развлечений. Разработка игр требует применения передовых технологий и концепций для создания увлекательного и захватывающего игрового контента.}
\abstractEN{The modern gaming industry is one of the most dynamically developing entertainment industries. Game development requires the use of advanced technologies and concepts to create engaging and exciting game content.}
\keysRU{Bomberman,проектирование,игра,unity,программирования}
\keysEN{Bomberman,design,game,unity,programming}

\date{\the\year}

% Подключение файлов с библиотекой.
\addbibresource{graduate-students.bib}

% Пакет для отладки отступов.
%\usepackage{showframe}

\begin{document}
\maketitle

\setcounter{page}{2}
\makeabstract
\tableofcontents

\chapter*{Введение}
\phantomsection\addcontentsline{toc}{chapter}{ВВЕДЕНИЕ}

В компьютерном веке, огромное количество людей имеют персональные компьютеры. Они используют их, как и для работы, так и для отдыха. Данное разработанное программное обеспечение как раз таки и охватывает сферу отдыха.

Актуальность данной работы обуславливается растущей индустрией в сфере компьютерных игр, а, следовательно, является привлекательной площадкой для разработчиков программного обеспечения и торговых компаний.

Цель данной работы --- создание интуитивно понятной игры «Bomberman», которая сочетала бы в себе простоту и увлекательный игровой процесс.

В компьютерном веке, огромное количество людей имеют персональные компьютеры. Они используют их, как и для работы, так и для отдыха.

Индустрия компьютерных игр зародилась в середине 1970-х годов как движение энтузиастов и за несколько десятилетий выросла из небольшого рынка в мейнстрим с годовой прибылью в 9.5 миллиардов долларов в США в 2007 году и 11.7 миллиардов в 2008 году (по отчетам ESA).

Современные персональные компьютеры продолжают вносить новшества в игровую индустрию. К значимым достижениям относятся улучшенные звуковые и графические карты, развитие операционных систем и центральных процессоров. В частности, операционные системы, такие как Windows 10 и macOS, обеспечивают более совершенное и безопасное игровое окружение, а современные процессоры, такие как Intel Core и AMD Ryzen, позволяют запускать игры с более высокой производительностью и графическим качеством.

В настоящее время игровая индустрия продолжает расти и развиваться, привлекая миллионы игроков по всему миру. Графика становится все более реалистичной, механики игр становятся более сложными, а виртуальная реальность и дополненная реальность открывают новые возможности для углубленного игрового опыта. Годовая прибыль от компьютерных игр продолжает увеличиваться, отражая рост популярности и значимости этой индустрии.

Современные игры --- одни из самых требовательных приложений на ПК. Многие мощные компьютеры покупаются геймерами, которые хотят играть в новейшие технологичные игры. Таким образом, игровая индустрия тесно связана с индустрией производства центральных процессоров, ведь игры зачастую требуют более высокой скорости работы процессора, чем бизнес-приложения.

Игра Bomberman –-- одна из самых старых головоломок, выпущенных в 80-х годах японской компанией Hudson Soft. По сути, это игра про симпатичного парня-робота в белом шлеме, который может производить бесконечное количество бомб и использовать их для уничтожения вещей и противников. В Европе игра известна под названием Eric and the Floaters.

Целью данной  работы является разработка игры Bomberman с использованием языка программирования Unity. В рамках этой цели были поставлены следующие задачи:

\begin{enumerate}
    \item Изучение основных концепций и механик игры Bomberman;
    \item Реализация игрового процесса, включая управление персонажем, генерацию уровней и логику взаимодействия с объектами;
    \item Проектирование игры включает определение игровых механик и графического стиля;
    \item Тестирование игры, выявление и исправление ошибок, балансировка игрового процесса;
    \item Получение опыта в разработке игр и применение полученных знаний и навыков.
\end{enumerate}

Путем выполнения этих задач, ожидается достижение качественного результата --- создание игры Bomberman, которая позволит пользователю насладиться захватывающим игровым процессом.
Таким образом, разработка игры Bomberman на языке Unity представляет собой интересное и актуальное задание, позволяющее погрузиться в мир разработки игр и применить полученные знания и навыки для создания увлекательного продукта.

% Подключение первой главы (теория):
\chapter{\label{ch:ch01}ГЛАВА 1. Теоритическая часть} % Нужно сделать главу в содержании заглавными буквами

\section{\label{sec:ch01/sec01} Общие сведения о компьютерной игре <<Bomberman>>}
Bomberman --- серия компьютерных игр и медиафранчайз, созданная компанией Hudson Soft. Первая игра серии Bomberman выпущена в 1983 году для домашних компьютеров и игровой консоли NES. В Европе игра распространялась под названием Eric and the Floaters. Игра стала началом одноименной серии. Игры серии разрабатывались и издавались как самой Hudson Soft, так и рядом других компаний и выходили на большинстве существующих игровых систем.

Большинство игр серии выполнены в жанре аркадного лабиринта. Игрок управляет персонажем, находящимся в лабиринте, состоящем из разрушаемых и неразрушаемых стен. Он может оставлять бомбу, взрывающуюся через некоторое фиксированное время и разрушающую стены рядом с ней. Специальные бонусы могут увеличить количество одновременно оставляемых бомб, дальность их взрыва, скорость перемещения героя, дать возможность взрыва бомб по нажатию кнопки, невосприимчивость от взрыва бомб, прохождение сквозь разрушаемые стены или сквозь собственные еще невзорванные бомбы. На уровне присутствуют противники. В некоторых играх серии целью игры является нахождение скрытой за одной из разрушаемых стен двери, ведущей в следующий уровень с предварительным уничтожением врагов. Другие игры рассчитаны на многопользовательскую игру на одном экране, целью в них является победа над всеми противниками ~\cite{wikiRUBomberman}.

\section{\label{sec:ch01/sec02} Кроссплатформенное программирование}
Кроссплатформенное программирование --- это методология разработки программного обеспечения, которая позволяет создавать приложения, которые могут работать на разных операционных системах и устройствах без необходимости переписывать код для каждой платформы отдельно.

Основная идея кроссплатформенного программирования заключается в том, чтобы использовать общий код, который может быть выполнен на разных платформах. Это позволяет разработчикам сэкономить время и усилия, так как им не нужно писать и поддерживать отдельные версии приложения для каждой платформы.

Кроссплатформенное программирование может быть осуществлено с использованием различных технологий и инструментов, таких как фреймворки, языки программирования и среды разработки. Некоторые из популярных технологий для кроссплатформенной разработки включают в себя React Native, Xamarin, Flutter и Cordova.

Кроссплатформенное программирование имеет ряд преимуществ, таких как повышение производительности разработки, снижение затрат на разработку и поддержку приложений, а также возможность достичь большей аудитории пользователей, работая на разных платформах.

Однако, кроссплатформенное программирование также имеет свои недостатки. Некоторые из них включают в себя ограничения в функциональности и возможностях, несовместимость с некоторыми платформами и устройствами, а также возможные проблемы с производительностью и оптимизацией.

В целом, кроссплатформенное программирование является важным инструментом для разработки мобильных и веб-приложений, которые должны работать на разных платформах. Оно позволяет разработчикам создавать эффективные и универсальные приложения, которые могут быть запущены на различных устройствах и операционных системах~\cite{wikiRUCrossplatform}.

\subsection{\label{subsec:ch01/sec02/sub02}Технологии и инструменты для кроссплатформенного программирования}

Кроссплатформенное программирование предоставляет разработчикам возможность создавать приложения, которые могут работать на разных операционных системах и устройствах. Для этого существует несколько технологий и инструментов, которые облегчают процесс разработки и обеспечивают совместимость с различными платформами.
\begin{itemize}
    \item \textbf{Фреймворки для мобильной разработки}. Одним из популярных подходов к кроссплатформенной разработке мобильных приложений является использование фреймворков, таких как React Native, Flutter и Xamarin. Эти фреймворки позволяют разработчикам писать код на одном языке программирования (например, JavaScript или Dart) и затем компилировать его в нативный код для разных платформ (iOS и Android). Это упрощает процесс разработки и позволяет создавать высокопроизводительные приложения с нативным интерфейсом.
    \item \textbf{Веб-технологии}. Другой подход к кроссплатформенной разработке --- использование веб-технологий, таких как HTML, CSS и JavaScript. С помощью фреймворков и инструментов, таких как React, Angular и Vue.js, разработчики могут создавать веб-приложения, которые могут работать на разных платформах и устройствах. Такие приложения могут быть упакованы в нативные оболочки с помощью инструментов, таких как Apache Cordova или Electron, чтобы они могли быть установлены и запущены как обычные приложения на разных платформах.
    \item \textbf{Нативное разработка с использованием языков программирования}. Еще один подход к кроссплатформенной разработке --- использование языков программирования, которые могут компилироваться в нативный код для разных платформ. Например, язык C++ с фреймворком Qt позволяет создавать кроссплатформенные приложения с нативным интерфейсом для разных операционных систем. Также существуют инструменты, такие как Unity и Unreal Engine, которые позволяют разрабатывать кроссплатформенные игры и визуализации.
\end{itemize}

Это лишь некоторые из технологий и инструментов, которые используются для кроссплатформенного программирования. Выбор конкретной технологии зависит от требований проекта, опыта разработчиков и особенностей платформ, на которых планируется запуск приложения ~\cite{book1author}.

\section{\label{sec:ch01/sec03} Обоснование и выбор ПО}
Unity --- это кросс-платформенный игровой движок и интегрированная среда разработки (IDE), разработанная компанией Unity Technologies. Unity позволяет разработчикам создавать игры и интерактивные приложения для различных платформ, включая компьютеры, мобильные устройства и игровые консоли.

Одной из ключевых особенностей Unity является его интуитивный визуальный редактор, который облегчает создание игровых сцен, объектов, компонентов и скриптов. Unity поддерживает использование различных языков программирования, включая C\#, JavaScript и Boo. Редактор Unity также предлагает широкий набор инструментов для работы с графикой, анимацией, физикой, звуком и другими аспектами разработки игр.

Unity имеет множество возможностей для создания сложных и качественных игровых проектов. Он поддерживает реализацию различных игровых жанров, включая 2D и 3D игры, а также предоставляет функциональность для работы с искусственным интеллектом, сетевым взаимодействием, виртуальной и дополненной реальностью.

Благодаря своей популярности и широкому сообществу разработчиков, Unity стал одним из наиболее популярных игровых движков в индустрии компьютерных игр. Он используется как независимыми разработчиками, так и крупными студиями для создания игр различных масштабов и жанров ~\cite{book5author,wikiRUUnity}.

\section{\label{sec:ch01/sec04} Обоснование и выбор технических средств}
Для разработки данной программы был использован компьютер со следующими характеристиками:
\begin{itemize}
    \item Процессор: Intel i5-12400F;
    \item Видеокарта: RTX 3050 на 8 GB;
    \item Память: ADATA XPG на 16 GB;
    \item SSD: ADATA SU650 на 1 TB;
    \item Операционная система Windows 10 Pro.
\end{itemize}
На основании предоставленных характеристик компьютера, можно сказать, что этот компьютер обладает достаточной мощностью для разработки программы в Unity.
Характеристики компьютера, такие как процессор Intel i5-12400F и видеокарта RTX 3050, являются достаточно современными и обеспечивают хорошую производительность при работе с Unity. 16 GB оперативной памяти и 1 TB SSD также являются достаточными для эффективной работы с проектами в Unity, позволяя быстро компилировать, запускать и изменять проекты.
Таким образом, данный компьютер с указанными характеристиками является приемлемым для разработки программы в Unity.


% Подключение второй главы (практическая часть):
\chapter{\label{ch:ch02}ГЛАВА 2. Практическая часть}

\section{\label{sec:ch02/sec01}Раздел 1}



\chapter*{Заключение}
\phantomsection\addcontentsline{toc}{chapter}{ЗАКЛЮЧЕНИЕ}
В ходе разработки игры Bomberman были изучены и применены различные концепции и технологии, связанные с разработкой игр. Этот проект представлял собой увлекательное и практическое погружение в мир создания игрового контента, позволяющее применить полученные навыки и знания.

Были определены основные игровые механики, которые обеспечивают захватывающий и динамичный игровой процесс. Внедрение управления персонажем, физики и генерации уровней позволило создать уникальное игровое пространство, которое предлагает разнообразие вызовов и стратегических возможностей для игрока.
Основной целью данного проекта было разработать игру Bomberman, которая будет востребована и позволит игрокам насладиться захватывающим игровым процессом. В результате выполнения поставленных задач и достижения качественного результата, цель проекта была успешно достигнута.

Разработка игры Bomberman на языке Unity предоставила возможность познакомиться с различными аспектами разработки игр, включая управление персонажем, физику, искусственный интеллект и графический дизайн. Этот проект также способствовал развитию навыков в области проектирования игровых механик, тестирования и отладки.

В целом, разработка игры Bomberman представляла собой захватывающее и практическое путешествие в мир разработки игр. Этот проект помог расширить понимание и применение концепций и технологий, связанных с разработкой игр, и внести свой вклад в развитие этой захватывающей и динамично развивающейся индустрии развлечений.

\newpage
\phantomsection\addcontentsline{toc}{chapter}{СПИСОК ИСПОЛЬЗОВАННОЙ ЛИТЕРАТУРЫ}
\printbibliography[title={Список использованной литературы}]

\appendix
\newpage
\chapter*{\raggedleft\label{appendix1}Приложение}
\phantomsection\addcontentsline{toc}{chapter}{ПРИЛОЖЕНИЕ}
%\section*{\centering\label{code:appendix}Текст программы}

\begin{center}
\label{code:appendix}Текст программы
\end{center}

\end{document}

