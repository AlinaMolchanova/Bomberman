\chapter{\label{ch:ch01}ГЛАВА 1. Теоритическая часть} % Нужно сделать главу в содержании заглавными буквами

\section{\label{sec:ch01/sec01} Общие сведения о компьютерной игре <<Bomberman>>}
Bomberman --- серия компьютерных игр и медиафранчайз, созданная компанией Hudson Soft. Первая игра серии Bomberman выпущена в 1983 году для домашних компьютеров и игровой консоли NES. В Европе игра распространялась под названием Eric and the Floaters. Игра стала началом одноименной серии. Игры серии разрабатывались и издавались как самой Hudson Soft, так и рядом других компаний и выходили на большинстве существующих игровых систем.

Большинство игр серии выполнены в жанре аркадного лабиринта. Игрок управляет персонажем, находящимся в лабиринте, состоящем из разрушаемых и неразрушаемых стен. Он может оставлять бомбу, взрывающуюся через некоторое фиксированное время и разрушающую стены рядом с ней. Специальные бонусы могут увеличить количество одновременно оставляемых бомб, дальность их взрыва, скорость перемещения героя, дать возможность взрыва бомб по нажатию кнопки, невосприимчивость от взрыва бомб, прохождение сквозь разрушаемые стены или сквозь собственные еще невзорванные бомбы. На уровне присутствуют противники. В некоторых играх серии целью игры является нахождение скрытой за одной из разрушаемых стен двери, ведущей в следующий уровень с предварительным уничтожением врагов. Другие игры рассчитаны на многопользовательскую игру на одном экране, целью в них является победа над всеми противниками ~\cite{wikiRUBomberman}.

\section{\label{sec:ch01/sec02} Кроссплатформенное программирование}
Кроссплатформенное программирование --- это методология разработки программного обеспечения, которая позволяет создавать приложения, которые могут работать на разных операционных системах и устройствах без необходимости переписывать код для каждой платформы отдельно.

Основная идея кроссплатформенного программирования заключается в том, чтобы использовать общий код, который может быть выполнен на разных платформах. Это позволяет разработчикам сэкономить время и усилия, так как им не нужно писать и поддерживать отдельные версии приложения для каждой платформы.

Кроссплатформенное программирование может быть осуществлено с использованием различных технологий и инструментов, таких как фреймворки, языки программирования и среды разработки. Некоторые из популярных технологий для кроссплатформенной разработки включают в себя React Native, Xamarin, Flutter и Cordova.

Кроссплатформенное программирование имеет ряд преимуществ, таких как повышение производительности разработки, снижение затрат на разработку и поддержку приложений, а также возможность достичь большей аудитории пользователей, работая на разных платформах.

Однако, кроссплатформенное программирование также имеет свои недостатки. Некоторые из них включают в себя ограничения в функциональности и возможностях, несовместимость с некоторыми платформами и устройствами, а также возможные проблемы с производительностью и оптимизацией.

В целом, кроссплатформенное программирование является важным инструментом для разработки мобильных и веб-приложений, которые должны работать на разных платформах. Оно позволяет разработчикам создавать эффективные и универсальные приложения, которые могут быть запущены на различных устройствах и операционных системах~\cite{wikiRUCrossplatform}.

\subsection{\label{subsec:ch01/sec02/sub02}Технологии и инструменты для кроссплатформенного программирования}

Кроссплатформенное программирование предоставляет разработчикам возможность создавать приложения, которые могут работать на разных операционных системах и устройствах. Для этого существует несколько технологий и инструментов, которые облегчают процесс разработки и обеспечивают совместимость с различными платформами.
\begin{itemize}
    \item \textbf{Фреймворки для мобильной разработки}. Одним из популярных подходов к кроссплатформенной разработке мобильных приложений является использование фреймворков, таких как React Native, Flutter и Xamarin. Эти фреймворки позволяют разработчикам писать код на одном языке программирования (например, JavaScript или Dart) и затем компилировать его в нативный код для разных платформ (iOS и Android). Это упрощает процесс разработки и позволяет создавать высокопроизводительные приложения с нативным интерфейсом.
    \item \textbf{Веб-технологии}. Другой подход к кроссплатформенной разработке --- использование веб-технологий, таких как HTML, CSS и JavaScript. С помощью фреймворков и инструментов, таких как React, Angular и Vue.js, разработчики могут создавать веб-приложения, которые могут работать на разных платформах и устройствах. Такие приложения могут быть упакованы в нативные оболочки с помощью инструментов, таких как Apache Cordova или Electron, чтобы они могли быть установлены и запущены как обычные приложения на разных платформах.
    \item \textbf{Нативное разработка с использованием языков программирования}. Еще один подход к кроссплатформенной разработке --- использование языков программирования, которые могут компилироваться в нативный код для разных платформ. Например, язык C++ с фреймворком Qt позволяет создавать кроссплатформенные приложения с нативным интерфейсом для разных операционных систем. Также существуют инструменты, такие как Unity и Unreal Engine, которые позволяют разрабатывать кроссплатформенные игры и визуализации.
\end{itemize}

Это лишь некоторые из технологий и инструментов, которые используются для кроссплатформенного программирования. Выбор конкретной технологии зависит от требований проекта, опыта разработчиков и особенностей платформ, на которых планируется запуск приложения ~\cite{book1author}.

\section{\label{sec:ch01/sec03} Обоснование и выбор ПО}
Unity --- это кросс-платформенный игровой движок и интегрированная среда разработки (IDE), разработанная компанией Unity Technologies. Unity позволяет разработчикам создавать игры и интерактивные приложения для различных платформ, включая компьютеры, мобильные устройства и игровые консоли.

Одной из ключевых особенностей Unity является его интуитивный визуальный редактор, который облегчает создание игровых сцен, объектов, компонентов и скриптов. Unity поддерживает использование различных языков программирования, включая C\#, JavaScript и Boo. Редактор Unity также предлагает широкий набор инструментов для работы с графикой, анимацией, физикой, звуком и другими аспектами разработки игр.

Unity имеет множество возможностей для создания сложных и качественных игровых проектов. Он поддерживает реализацию различных игровых жанров, включая 2D и 3D игры, а также предоставляет функциональность для работы с искусственным интеллектом, сетевым взаимодействием, виртуальной и дополненной реальностью.

Благодаря своей популярности и широкому сообществу разработчиков, Unity стал одним из наиболее популярных игровых движков в индустрии компьютерных игр. Он используется как независимыми разработчиками, так и крупными студиями для создания игр различных масштабов и жанров ~\cite{book5author,wikiRUUnity}.

\section{\label{sec:ch01/sec04} Обоснование и выбор технических средств}
Для разработки данной программы был использован компьютер со следующими характеристиками:
\begin{itemize}
    \item Процессор: Intel i5-12400F;
    \item Видеокарта: RTX 3050 на 8 GB;
    \item Память: ADATA XPG на 16 GB;
    \item SSD: ADATA SU650 на 1 TB;
    \item Операционная система Windows 10 Pro.
\end{itemize}
На основании предоставленных характеристик компьютера, можно сказать, что этот компьютер обладает достаточной мощностью для разработки программы в Unity.
Характеристики компьютера, такие как процессор Intel i5-12400F и видеокарта RTX 3050, являются достаточно современными и обеспечивают хорошую производительность при работе с Unity. 16 GB оперативной памяти и 1 TB SSD также являются достаточными для эффективной работы с проектами в Unity, позволяя быстро компилировать, запускать и изменять проекты.
Таким образом, данный компьютер с указанными характеристиками является приемлемым для разработки программы в Unity.

